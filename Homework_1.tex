\documentclass{article}
\usepackage{267_Lib}

\title{MATH 267 - Homework 1}
\author{Sele Okojie}
\date{February 6, 2023}

\begin{document}
    \maketitle
    
    \begin{enumerate}

            %----------------------------------------Question 1----------------------------------------%
    	\item Prove each of the following statements:
    		\begin{enumerate}
      
                    %---------1a---------%
    			\item There exists an integer $x$ such that $x \cdot 3 = 12$.
                        \begin{proof}
                            Let $x = 4 \in \mathbb{Z}$. Then, $x \cdot 3 = 4 \cdot 3 = 12$.
                        \end{proof}

                    %---------1b---------%
    			\item There exists an integer $x$ such that $x + 10 = x \cdot 3$.
                        \begin{proof}
                            Let $x = 5 \in \mathbb{Z}$. Then, $x + 10 = 5 + 10 = 15 = 5 * 3 = x * 3$.
                        \end{proof}

                    %---------1c---------%
    			\item For all integers $x$, $x + 3 \le x + 7$.
                        \begin{proof}
                            Fix an arbitrary integer $x$. We know that $3 \leq 7$. Therefore,
                            \begin{align*}
                                3 + x &\le 7 + x            \qquad\textrm{by Axiom 3b}      \\  %We need the '\\' to note the end of the line (Not at the final line though)
                                x + 3 &\le x + 7            \qquad\textrm{by Axiom 1a.} %The '&=' tells the package where the equals sign is for alignment
                            \end{align*}
                        \end{proof}

                    %---------1d---------%
    			\item For all integers $x$, if $x \ge 0$, then $x \cdot 3 \ge 0$.
                        \begin{proof}
                            Fix an arbitrary integer $x$ and assume $x \ge 0$. Then, since $x \ge 0$ and $3 \ge 0$,
                            \begin{alignat*}{2}
                                x \cdot 3 &\ge 0 \cdot 2    &\qquad &\textrm{by Axiom 3c}   \\
                                3 \cdot x &\ge 0            &\qquad &\textrm{by Axiom 2a.}
                            \end{alignat*}
                        \end{proof}

                    %---------1e---------%
    			\item For all integers $x$, $0 + x = 1 \cdot x$.
                        \begin{proof}
                            Fix an arbitrary integer $x$. Then,
                            \begin{alignat*}{2}
                                x &= x \\
                                x + 0 &= x                  &\qquad &\textrm{by Axiom 1c}   \\
                                x + 0 &= x \cdot 1          &\qquad &\textrm{by Axiom 2c}   \\
                                0 + x &= 1 \cdot x          &\qquad &\textrm{by Axioms 1a and 2a.}
                            \end{alignat*}
                        \end{proof}

                    %---------1f---------%
    			\item For all integers $x$, there exists an integer $y$ such that $x + y = 17$.
                        \begin{proof}
                            Fix an arbitrary integer $x$. Let $y = (17 + (\minus x)) \in \mathbb{Z}$. Then, 
                            \begin{alignat*}{2}
                                x + y &= x + (17 + (\minus x)) \\
                                &= x + ((\minus x ) + 17)   &\qquad &\textrm{by Axiom 1a}   \\
                                &= (x + (\minus x)) + 17    &\qquad &\textrm{by Axiom 1b}   \\
                                &= 0 + 17                   &\qquad &\textrm{by Axiom 1d}   \\
                                &= 17                       &\qquad &\textrm{by Axiom 1c.}  
                            \end{alignat*}
                        \end{proof}

                    %---------1g---------%
    			\item There exists an integer $x$ such that, for all integers $y$, $y + x = y$.
                        \begin{proof}
                            Fix an arbitrary integer $y$. Let $x = 0 \in \mathbb{Z}$. Then,
                            \begin{alignat*}{2}
                                x &= y + (\minus y)         &\qquad &\textrm{by Axiom 1d.}  \\
                                \textrm{Hence, } x + y &= y + (\minus y) + y &\qquad &\textrm{by adding '$y$' to both sides}\\
                                x + y &= y + 0              &\qquad &\textrm{by Axiom 1d}  \\
                                x + y &= y                  &\qquad &\textrm{by Axiom 1c}  \\
                                y + x &= y                  &\qquad &\textrm{by Axiom 1a.}  
                            \end{alignat*}
                        \end{proof}
                        
    		\end{enumerate}

            %----------------------------------------Question 2----------------------------------------%
    	\item Prove that each of the following statements are false:
    		\begin{enumerate}

                    %---------2a---------%
    			\item For all integers $x$, $x + 3 = 4$. \\\\
                    We must prove that: ``There exists an integer $x$ such that $x + 3 \ne 4$".
                        \begin{proof}
                            Let $x = 2 \in \mathbb{Z}$. Then, $x + 3 = 2 + 3 = 5 \ne 4$. 
                        \end{proof}
                    Since the negation was proven, the original statement must be false.

                    %---------2b---------%
    			\item There exists an integer $x$ such that $x > 0$ and $x < 0$. \\\\
                    We must prove that: ``For all integers $x$, $x \le 0$ or $x \ge 0$" $\equiv$\\ ``For all integers $x$, if $x > 0$, then $x \ge 0$." 
                        \begin{proof}
                            Fix an arbitrary integer $x$ and assume $x > 0$. Then, $x \ge 0 \equiv (x > 0$ or $ x = 0)$ . Therefore, $x \ge 0$.
                        \end{proof}
                    Since the negation was proven, the original statement must be false.

                    %---------2c---------%
    			\item There exists an integer $x$ such that, for all integers $y$, $x + y = 17$. \\\\
                    We must prove that: ``For all integers $x$, there exists an integer $y$ such that $x + y = 17$." Note that this was proved in (1f). 
                        \begin{proof}
                            Fix an arbitrary integer $x$. Let $y = (17 + (\minus x)) \in \mathbb{Z}$. Then, 
                            \begin{alignat*}{2}
                                x + y &= x + (17 + (\minus x)) \\
                                &= x + ((\minus x ) + 17)   &\qquad &\textrm{by Axiom 1a}   \\
                                &= (x + (\minus x)) + 17    &\qquad &\textrm{by Axiom 1b}   \\
                                &= 0 + 17                   &\qquad &\textrm{by Axiom 1d}   \\
                                &= 17                       &\qquad &\textrm{by Axiom 1c.}  
                            \end{alignat*}
                        \end{proof}
                        Since the negation was proven, the original statement must be false.

    		\end{enumerate}

            %----------------------------------------Question 3----------------------------------------%
    	\item Prove that, for all integers $x$, if $x \ge 0$, then $\minus x \le 0$.
                \begin{proof}
                    Fix an arbitrary integer $x$ and assume that $x \ge 0$. Then,
                    \begin{alignat*}{2}
                        0 &\le x                             &\qquad &\textrm{by definition.} \\
                        \textrm{Hence, } 0 + (\minus x) &\le x + (\minus x) &\qquad &\textrm{by adding '$\minus x$' to both sides}  \\
                        (\minus x) + 0 &\le x + (\minus x)  &\qquad &\textrm{by Axiom 1a}   \\
                        \minus x &\le x + (\minus x)        &\qquad &\textrm{by Axiom 1c}  \\
                        \minus x &\le 0                     &\qquad &\textrm{by Axiom 1d.}  \\
                    \end{alignat*}
                \end{proof}
     
    \end{enumerate}
    
    \newpage
    \textbf{Bonus Questions:}
    
    \begin{itemize}

            %---------B1---------%
    	\item [(B1)] Forget everything you know about $0$ and $1$.  The integer $0$ comes from Axiom 1.2.1 (1)(c) and the integer $1$ comes from Axiom     1.2.1 (2)(c).

    	Formally prove, using only Axiom 1.2.1, that $0 \le 1$.
                \begin{proof}
                    Idk :(
                \end{proof}
            %---------B2---------%
    	\item [(B2)] Prove that, for all integers $x$, $x \cdot x \ge 0$.  (In other words, ``The square of every integer is non-negative.'')
            \begin{proof}
                Fix an arbitrary integer $x$. According to Trichotomy (Axiom 3a), $x \le 0$ or $x \ge 0$. We can split this into 3 cases: $x = 0$, $x < 0$, and $x > 0$.\\
                
                \emph{Case 1: $x = 0$}: Let $y = 0 \cdot 0$. Then, 
                    \begin{alignat*}{2}
                        y &= x \cdot x \\
                        &= x \cdot (x + x) \\
                        &= x \cdot x + x \cdot x \\
                        &= y + y
                    \end{alignat*}
                    Thus, $y = y + y$ and
                    \begin{alignat*}{2}
                        x &= y + (\minus y)             &\qquad &\textrm{by Axiom 1d}  \\
                        &= y + y + (\minus y) = y           &\qquad &\textrm{since $y = y + y$} 
                    \end{alignat*} 
                    Thus, $0 \cdot 0 = 0$, which is non-negative.\\

                    \emph{Case 2: $x < 0$}: Fix two arbitrary positive integers $a$ and $b$. Then, 
                    \begin{alignat*}{2}
                        (\minus a)\cdot b + a\cdot b &= (\minus a)\cdot b + a\cdot b \\
                        (\minus a)\cdot b + a\cdot b &= b\cdot (a + (\minus a)) \\
                        (\minus a)\cdot b + a\cdot b &= b\cdot (0) \\
                        (\minus a)\cdot b + a\cdot b &= 0 \\
                        (\minus a)\cdot b &= \minus a\cdot b \\
                        (\minus a)\cdot (\minus b) + (\minus a\cdot b) &= (\minus a)\cdot (\minus b) + (\minus a)\cdot b \\
                        (\minus a)\cdot (\minus b) + (\minus a\cdot b) &= (\minus a)\cdot (b - b) \\
                        (\minus a)\cdot (\minus b) + (\minus a\cdot b) &= (\minus a)\cdot (0) \\
                        (\minus a)\cdot (\minus b) &= a\cdot b
                    \end{alignat*}
                    Thus, the product of two arbitrary negative integers is positive. \\

                    \emph{Case 3: $x > 0$}: Fix an arbitrary integer $a$ such that $a \ge 0 \equiv 0 \le a$. Thus,
                    \begin{alignat*}{2}
                        a \cdot 0 &\le a \cdot a        &\qquad &\textrm{by Axiom 3c}   \\
                        0 &\le a \cdot a                &\qquad &\textrm{by Lemma}
                        %Note, we need to add the Lemma about a * 0 = 0
                    \end{alignat*}
                    Thus, the product of two arbitrary positive integers is positive \\\\
                    Since in all cases of squaring result in a non-negative integer, the square of every integer is non-negative.
            \end{proof}
    \end{itemize}
    
\end{document}
