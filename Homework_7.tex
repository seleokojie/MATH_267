\documentclass{article}
\usepackage{267_Lib}

\title{MATH 267 - Homework 7}
\author{Sele Okojie}
\date{March 27, 2023}

\begin{document}
    \maketitle

    \begin{enumerate}

        %---------------Question 1---------------%
	\item Prove that the rationals are dense in themselves.

        %---------------Question 2---------------%
	\item Prove that the integers are complete.

        %---------------Question 3---------------%
	\item For each of the following, determine if it is a function or not.
		\begin{align*}
			f = & \ \{ (1, 2), (2, 3), (3, 4), (4, 5) \} \\
			g = & \ \{ (1, 2), (1, 3), (1, 4), (1, 5) \} \\
			h = & \ \{ (1, 5), (2, 5), (3, 5), (4, 5) \} \\
			i = & \ \{ (x,x) : x \in \mathbb{Z} \} \\
			j = & \ \{ (x,y) : x, y \in \mathbb{R} \text{ and } x < y \} \\
			k = & \ \{ (x,y) : x, y \in \mathbb{R} \text{ and } x + y = 17 \}
		\end{align*}

        %---------------Question 4---------------%
	\item For each function in (3), find its domain and image.

        %---------------Question 5---------------%
	\item Determine if the following is a well-defined function $f : \mathbb{Q} \rightarrow \mathbb{Q}$.  Justify your answer.
		\[
			f \left( \frac{p}{q} \right) = \frac{ p + q }{ q^3 }.
		\]

        %---------------Question 6---------------%
	\item Determine if the following is a well-defined function $f : \mathbb{Q} \rightarrow \mathbb{Q}$.  Justify your answer.
		\[
			f \left( \frac{p}{q} \right) = \frac{ 2p + q }{ 3q }.
		\]

        %---------------Question 7---------------%
	\item For each function below, determine its image.
		\begin{align*}
			& f : \mathbb{R} \rightarrow \mathbb{R} & & f(x) = x^2 \\
			& g : (\mathbb{R} \setminus \{ 3 \}) \rightarrow \mathbb{R} & & g(x) = \frac{2x + 5}{x - 3} \\
			& h : \mathbb{R}^2 \rightarrow \mathbb{R} & & h((x,y)) = x - y \\
			& i : \mathbb{R} \rightarrow \mathbb{R}^2 & & i(x) = (x, 2x) 
		\end{align*}

        %---------------Question 8---------------%
	\item Justify your answer for $g$ in question (7).
    \end{enumerate}

    \newpage
    \textbf{Bonus Questions:}
    
    \begin{itemize}

        %---------------Question B1---------------%
        \item [(B1)] How many functions are there with domain $\{ 1, 2, 3, 4 \}$ and codomain $\{ 1, 2, 3 \}$?

        %---------------Question B2---------------%
        \item [(B2)] Give an example of a function $f : \mathbb{R} \rightarrow \mathbb{R}^2$ whose image is the unit circle.  In other words, $\mathrm{im}(f) = \left\{ (x,y) : x, y \in \mathbb{R} \text{ and } x^2 + y^2 = 1 \right\}$ = ?
    \end{itemize}

\end{document}