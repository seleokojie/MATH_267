\documentclass{article}
\usepackage{267_Lib}

\title{MATH 267 - Homework 8}
\author{Sele Okojie}
\date{April 3, 2023}

\begin{document}
    \maketitle

    \begin{enumerate}

        %---------------Question 1---------------%
	\item For each function below, determine if it is injective and determine if it is surjective:
		  \begin{align*}
			f : & \ \mathbb{R} \rightarrow \mathbb{R} & f(x) = & \ 7 - x^3 \\
			g : & \ \mathbb{R} \rightarrow \mathbb{R} & g(x) = & \ x^2 + 1 \\
			h : & \ \mathbb{Z} \rightarrow \mathbb{Z} & h(n) = & \ 3n \\
			i : & \ \{ 1, 2 \} \rightarrow \{ 1, 2, 3 \} & i(1) = & \ 2, \ i(2) = 3 \\
			j : & \ \{ 1, 2, 3 \} \rightarrow \{ 1, 2 \} & j(1) = & \ 3, \ j(2) = 2, \ j(3) = 1
		  \end{align*}

            \begin{itemize}
                \item $f$ is both injective and surjective.
                \item $g$ is not injective $\big( g(1) = g(\minus 1) = 2 $, but $1\ne\minus1\big)$ nor surjective $\big($Suppose $b = 0\in\mathbb{R}$. There is no $a\in\mathbb{R}$ such that $g(a) = b\big)$.
                \item $h$ is injective but not surjective $\big($Suppose $b = 1\in\mathbb{Z}$. There is no $a\in\mathbb{Z}$ such that $h(a) = b\big)$.
                \item $i$ is injective but not surjective $\big(1\in\{ 1, 2, 3 \}$, but there is no $a\in\{ 1, 2 \}$ such that $i(a) = 1\big)$.
                \item $j$ is not a valid function, so we cannot determine if $j$ is injective or surjective.
            \end{itemize}
        
        %---------------Question 2---------------%
	\item Prove that the following function is not injective and not surjective:
		  \begin{align*}
			f : & \ \mathbb{R} \rightarrow \mathbb{R} & f(x) = & \ |x|
		  \end{align*}

            \begin{proof}We will show that $f$ is neither injective nor surjective.

                \begin{caseof}
                    \case{(Not Injective)}{
                        Let $a_1 = 1\in\mathbb{R}$ and $a_2 = \minus 1\in\mathbb{R}$. Then, $f(a_1) = f(1) = 1$ and $f(a_2) = f(\minus 1) = 1$. So, $f(a_1) = f(a_2) = 1$. However, $a_1 \ne a_2$. \\ Thus, $f$ is not injective.
                    }\case{(Not Surjective)}{
                        Note that the codomain of f is $\mathbb{R}$. However, f(x) is always non-negative for all $a\in\mathbb{R}$. Hence, there is no real number $a$ such that $f(a) < 0$. Thus, $f$ is not surjective.
                    }
                \end{caseof}
                Therefore, $f$ is neither injective nor surjective.
            \end{proof}

        %---------------Question 3---------------%
	\item Prove that the following function is injective but not surjective:
		  \begin{align*}
			f : & \ \mathbb{Z} \rightarrow \mathbb{Z} & f(n) = & \ 2n+1
		  \end{align*}

            \begin{proof}We will show that $f$ is injective but not surjective.

                \begin{caseof}
                    \case{(Injective)}{
                        Fix arbitrary $n_1, n_2 \in\mathbb{Z}$ and assume that $f(n_1) = f(n_2)$. Then,
                        \begin{align*}
                            2n_1 + 1 &= 2n_2 + 1 \\
                            2n_1 &= 2n_2 \\
                            n_1 &= n_2.
                        \end{align*}
                        Thus, $f$ is injective.
                    }\case{(Not Surjective)}{
                        Let $b = 2\in\mathbb{Z}$. Towards a contradiction, we will assume there exists some $n\in\mathbb{Z}$ such that $f(n) = b$. Then, $2n + 1 = 2$ and $n = \frac{1}{2}$. However, $n$ is an integer, leading to a contradiction. Thus, we reject our assumption that this $n$ exists and conclude that $f$ is not surjective.
                    }
                \end{caseof}
                Therefore, $f$ is injective but not surjective.
            \end{proof}

        %---------------Question 4---------------%
	\item For each function below, determine if it is bijective or not.  If it is bijective, provide the inverse function.  If it is not bijective, prove it is not:
		  \begin{align*}
			f : & \ \{ 1, 2, 3, 4 \} \rightarrow \{ 2, 4, 6, 8 \} & f(n) = & \ 10 - 2n \\
			g : & \ \mathbb{Z} \rightarrow \mathbb{Z} & g(n) = & \ 10 - 2n \\
			h : & \ \mathbb{R} \rightarrow \mathbb{R} & h(x) = & \ x^3 + 1 \\
			i : & \ \mathbb{R} \rightarrow \mathbb{R} & i(x) = & \ x^2 - 2x + 1 \\
			j : & \ \mathbb{R} \setminus \{ 1 \} \rightarrow \mathbb{R} \setminus \{ 0 \} & j(x) = & \ \frac{5}{x-1} \\
			k : & \ \mathbb{R}^2 \rightarrow \mathbb{R}^2 & k((x,y)) = & \ (2x+y, x-y) \\
			l : & \ \mathbb{R} \rightarrow \mathbb{R}^2 & l(x) = & \ (\cos(x), \sin(x))
		  \end{align*}

            \begin{itemize}
                \item $f$ is bijective. Its inverse function is: 
                    \begin{align*}
                        f^{\minus1} : & \ \{ 2, 4, 6, 8 \} \rightarrow \{ 1, 2, 3, 4 \} & f^{\minus1}(n) = & \ \frac{10 - n}{2}.
                    \end{align*}

                \item $g$ is not bijective because it is not surjective.
                    \begin{proof}
                        Let $b = 1\in\mathbb{Z}$. Towards a contradiction, we will assume there exists some $n\in\mathbb{Z}$ such that $g(n) = b$. Then, $10 - 2n = 1$ and $n = \frac{9}{2}$. However, $n$ is an integer, leading to a contradiction. Thus, we reject our assumption that this $n$ exists and conclude that $g$ is not surjective. 
                    \end{proof}

                \item $h$ is bijective. Its inverse function is:
                    \begin{align*}
                        h^{\minus1} : & \ \mathbb{R} \rightarrow \mathbb{R} & h^{\minus1}(x) = & \ \sqrt[3]{x - 1}.
                    \end{align*}
                    
                \item $i$ is not bijective because it is not surjective.
                    \begin{proof}
                        Let $b = \minus1\in\mathbb{Z}$. Towards a contradiction, we will assume there exists some $n\in\mathbb{Z}$ such that $i(x) = b$. Then,
                        \begin{align*}
                            x^2 - 2x + 1 &= \minus 1 \\
                            (x - 1)^2 &= \minus 1.
                        \end{align*}
                        Since a square is always non-negative, this is a contradiction. Thus, we reject our assumption that this $x$ exists and conclude that $i$ is not surjective. 
                    \end{proof}
                    
                \item $j$ is bijective. Its inverse function is:
                    \begin{align*}
                        j^{\minus1} : & \ \mathbb{R} \setminus \{ 0 \} \rightarrow \mathbb{R} \setminus \{ 1 \} & j^{\minus1}(x) = & \ \frac{5}{x} + 1.
                    \end{align*}

                \item $k$ is bijective. Its inverse function is:
                \begin{align*}
                    k^{\minus1} : & \ \mathbb{R}^2 \rightarrow \mathbb{R}^2 & k^{\minus1}((x,y)) = & \ \left(\frac{x + y}{3}, \frac{x - 2y}{3} \right)
                \end{align*}

                \item $l$ is not bijective because it is not injective.
                    \begin{proof}
                        Let $a_1 = 0$ and $a_2 = 2\pi$. Then, $f(a_1) = f(0) = (1, 0)$ and $f(a_2) = f(2\pi) = (1, 0)$. So, $f(a_1) = f(a_2)$, but $a_1 \ne a_2$. Thus, $f$ is not injective.
                    \end{proof}
            \end{itemize}

        %---------------Question 5---------------%
	\item Give an example of a bijection from $[0,1]$ to $[2,5]$.
            \[
                f(x) = 3x + 2.
            \]
    \end{enumerate}

    \newpage
    \textbf{Bonus Questions:}

    \begin{itemize}

        %---------------Question B1---------------%
	\item [(B1)] \textit{Generalize Question 5}. Prove that, for all real numbers $a$, $b$, $c$, and $d$, if $a < b$ and $c < d$, then there exists a bijection $f : [a,b]     \rightarrow [c,d]$.
            \begin{proof}
                Let $f : [a, b] \rightarrow [c, d]$ be defined as:
                \begin{align*}
                    f(x) = (x - a) \cdot \frac{d - c}{b - a} + c
                \end{align*}
                for all $x\in[a, b]$. We will show that f is a bijection, meaning that it is both injective and surjective.
                \begin{caseof}
                    \case{(Injective)}{
                        Fix arbitrary $x, y \in\mathbb{Z}$ and assume that $f(x) = f(y)$. Then,
                        \begin{align*}
                            (x - a) \cdot \frac{d - c}{b - a} + c &= (y - a) \cdot \frac{d - c}{b - a} + c \\
                            (x - a) \cdot \frac{d - c}{b - a} &= (y - a) \cdot \frac{d - c}{b - a}.
                        \end{align*}
                        Since $c < d$ and $a < b$, $(d - c) > 0$ and $(b - a) > 0$. Thus, we can divide the common fraction out and get:
                        \begin{align*}
                            x - a &= y - a \\
                            x &= y.
                        \end{align*}
                        Thus, $f$ is injective.
                    }\case{(Surjective)}{
                        Fix some $n\in\mathbb{R}$ and let $m = (n - c) \cdot \frac{b - a}{d - c} + a$. Then,
                        \begin{align*}
                            f(m) = (m - a) \cdot \frac{d - c}{b - a} + c &= \left((n - c) \cdot \frac{b - a}{d - c} + a - a\right) \cdot \frac{d - c}{b - a} + c \\
                            &= (n - c) \cdot \frac{b - a}{d - c} \cdot \frac{d - c}{b - a} + c.
                        \end{align*}
                        Since $c < d$ and $a < b$, $(d - c) > 0$ and $(b - a) > 0$. Thus, we can cancel the common terms out and get:
                        \begin{align*}
                            f(m) &= n - c + c \\
                            &= n.
                        \end{align*}
                        Thus, $f$ is surjective.
                    }
                \end{caseof}
                Since $f$ is both injective and surjective, then, for all real numbers $a$, $b$, $c$, and $d$, if $a < b$ and $c < d$, then there exists a bijection $f : [a,b]     \rightarrow [c,d]$.
            \end{proof}

        %---------------Question B2---------------%
	\item [(B2)] Find a injective function from $\mathbb{Z} \times \mathbb{Z}$ to $\mathbb{Q}$.
            \[
                f(m, n) = 2^m3^n.
            \]
    \end{itemize}

\end{document}