\documentclass{article}
\usepackage{267_Lib}

\title{MATH 267 - Homework 9}
\author{Sele Okojie}
\date{April 10, 2023}

\begin{document}
    \maketitle

    \begin{enumerate}

        %---------------Question 1---------------%
        \item Prove that, for all sets $A$, $\mathrm{id}_A \circ \mathrm{id}_A = \mathrm{id}_A$.
            \begin{proof}
                Fix an arbitrary set $A$ and an arbitrary $a\in A$. Then,
                \[
                    \mathrm{id}_A \circ \mathrm{id}_A = (\mathrm{id}_A \circ \mathrm{id}_A)(a) = \mathrm{id}_A(\mathrm{id}_A(a)) = \mathrm{id}_A(a) = \mathrm{id}_A.
                \]
            \end{proof}

        %---------------Question 2---------------%
	\item Let $A = \{ 1, 2, 3 \}$, $B = \{ 4, 5, 6 \}$, and $f : A \rightarrow B$ be given by
		\[
			f(1) = 5, \ f(2) = 5, \ f(3) = 6.
		\]
		Compute each of the following images and preimages:
		\begin{align*}
			\text{(a)} & \ \ f(\{ 1 \}) = \{ 5 \} & \text{(b)} & \ \ f(\{ 1, 2 \}) = \{ 5 \} \\
			\text{(c)} & \ \ f( \emptyset ) = \emptyset & \text{(d)} & \ \ f^{-1}(\{ 4 \}) = \emptyset \\
			\text{(e)} & \ \ f^{-1}(\{ 5 \}) = \{ 1, 2 \} & \text{(f)} & \ \ f^{-1}(\{ 4, 6 \}) = \{ 3 \}
		\end{align*}

        %---------------Question 3---------------%
	\item Let $f : \mathbb{R} \rightarrow \mathbb{R}$ be given by
		\[
			f(x) = x^3 - 3x.
		\]
		Compute each of the following images and preimages:
		\begin{align*}
			\text{(a)} & \ \ f(\{0, 1\}) = \{ 0, \minus2 \} & \text{(b)} & \ \ f([0,1]) = [\minus2, 0] \\
			\text{(c)} & \ \ f(\emptyset) = \emptyset & \text{(d)} & \ \ f^{-1}(\{ 2 \}) = \{ 2 \} \\
			\text{(e)} & \ \ f^{-1}([-2,2]) = [\minus2, 2] & \text{(f)} & \ \ f^{-1}(\emptyset) = \emptyset
		\end{align*}

        %---------------Question 4---------------%
	\item Let $f : \mathbb{R} \rightarrow \mathbb{R}^2$ be given by
		\[
			f(x) = (\cos(x), \sin(x)).
		\]
		Draw each of the following images in the Euclidean plane:
		\begin{align*}
			\text{(a)} & \ \ f(\{ 0 \}) & \text{(b)} & \ \ f([0, \pi])
		\end{align*}

        %---------------Question 5---------------%
	\item \emph{Proposition 10.2.1 (2)}: Prove that, for all sets $A$ and $B$, for all $C_1, C_2 \subseteq A$, and for all functions $f : A \rightarrow B$,
		\[
			f(C_1 \cap C_2) \subseteq f(C_1) \cap f(C_2).
		\]

        %---------------Question 6---------------%
	\item \emph{Proposition 10.2.1 (8)}: Prove that, for all sets $A$ and $B$, for all $D \subseteq B$, and for all functions $f : A \rightarrow B$,
		\[
			f(f^{-1}(D)) \subseteq D.
		\]
    \end{enumerate}

\newpage
\textbf{Bonus Questions:}

\begin{itemize}
	\item [(B1)] \emph{What assumptions do we really need to prove Theorem 10.1.3?}  Assume that $X$ is a set and $*$ and ${}^{-1}$ are operations on $X$ such that,
		\begin{itemize}
			\item for all $a, b \in X$, $a * b \in X$,
			\item for all $a \in X$, $a^{-1} \in X$,
			\item for all $a, b, c \in X$, $a * (b * c) = (a * b) * c$,
			\item there exists $e \in X$ such that $a * e = e * a = a$,
			\item for all $a \in X$, $a * a^{-1} = a^{-1} * a = e$.
		\end{itemize}
		Prove that, for all $a, b \in X$, $(a * b)^{-1} = b^{-1} * a^{-1}$.
	\item [(B2)] Consider the function $f : \mathbb{R}^2 \rightarrow \mathbb{R}$ given by
		\[
			f((x,y)) = x + y.
		\]
		Compute $f^{-1}( \{ 0 \} )$ and justify your answer.
\end{itemize}

\end{document}