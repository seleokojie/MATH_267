\documentclass{article}
\usepackage{267_Lib}

\title{MATH 267 - Homework 4}
\author{Sele Okojie}
\date{February 27, 2023}

\begin{document}
    \maketitle

    \begin{enumerate}
    
        %---------------Question 1---------------%
	\item \emph{Ordered versus unordered pairs}: Give an example of integers $a$, $b$, $c$, and $d$ such that $\{ a, b \} = \{ c, d \}$ but $(a,b) \neq (c,d)$.\\\\
            Let $a = 1$, $b = 2$, $c = 2$, and $d = 1$. Then, substituting into the original equations, $\{ 1, 2 \} = \{ 2, 1 \}$ but $(1,2) \neq (2,1)$.
        
        %---------------Question 2---------------%
	\item \emph{Cartesian product is not commutative}: Give an example of sets $A$ and $B$ such that $A \times B \neq B \times A$.
            Let $A = \{ 1, 2 \}$ and $B = \{ 1, 2, 3 \}$. Then,
            \begin{alignat*}{2}
                A \times B = \{ (1, 1), (1, 2), (1, 3), (2, 1), (2, 2), (2, 3) \} \\
                B \times A = \{ (1, 1), (1, 2), (2, 1), (2, 2), (3, 1), (3, 2) \}
            \end{alignat*}
            Thus, $A \times B \neq B \times A$.
	
        %---------------Question 3---------------%
        \item Prove that, for all sets $A$, $B$, and $C$, $A \times (B \cup C) = (A \times B) \cup (A \times C)$.
            \begin{proof}
                Fix arbitrary sets $A$, $B$, and $C$. $A \times (B \cup C) = (A \times B) \cup (A \times C)$ means: 
                    \begin{align*}
                        \text{i. } &A \times (B \cup C) \subseteq (A \times B) \cup (A \times C) \\
                        \text{ii. } &(A \times B) \cup (A \times C) \subseteq A \times (B \cup C).
                    \end{align*}
                    
                \begin{enumerate}[i.]
                    \item \lQuote $A \times (B \cup C) \subseteq (A \times B) \cup (A \times C)$" means:
                        \begin{align*}
                            \lQuote \text{For all } z \in A \times (B \cup C)\text{, } z \in (A \times B) \cup (A \times C)\text{."}
                        \end{align*}
                        
                         To prove this, we must fix an arbitrary $z \in A \times (B \cup C)$.\ Then, $z = (x, y)$, $x \in A$ and $y \in (B \cup C)$. So, $y \in B$ or $y \in C$. Therefore, $(x \in A$ and $y \in B)$ or $(x \in A$ and $y \in C)$. Thus, 
                         \begin{alignat*}{2}
                             &z = (x,y) \in (A \times B) \cup (A \times C) \text{ and} \\
                             &z \in (A \times B) \cup (A \times C)\text{.}
                         \end{alignat*}
                         
                     \item \lQuote $(A \times B) \cup (A \times C) \subseteq A \times (B \cup C)$" means:
                        \begin{align*}
                            \lQuote \text{For all } z \in (A \times B) \cup (A \times C)\text{, } z \in A \times (B \cup C)\text{."}
                        \end{align*}

                        To prove this, we must fix an arbitrary $z \in (A \times B) \cup (A \times C)$. Therefore, $z \in (A \times B)$ or $z \in (A \times C)$.
                        \begin{caseof}
                            \case{$z \in (A \times B)$}{
                                Then, z = $(x, y)$, $x \in A$ and $y \in B$. So, $y \in (B \cup C)$. Thus,
                                \begin{alignat*}{2}
                                    &z = (x,y) \in A \times (B \cup C) \text{ and} \\
                                    &z \in A \times (B \cup C)\text{.}
                                \end{alignat*}
                
                            }\case{$z \in (A \times C)$}{
                                Then, z = $(x, y)$, $x \in A$ and $y \in C$. So, $y \in (B \cup C)$. Thus,
                                \begin{alignat*}{2}
                                    &z = (x,y) \in A \times (B \cup C) \text{ and} \\
                                    &z \in A \times (B \cup C)\text{.}
                                \end{alignat*}
                            }
                        \end{caseof}
                \end{enumerate}

                Therefore, since 
                \begin{alignat*}{2}
                    &A \times (B \cup C) \subseteq (A \times B) \cup (A \times C)\text{ and} \\
                    &(A \times B) \cup (A \times C) \subseteq A \times (B \cup C)\text{, then} \\
                    &A \times (B \cup C) = (A \times B) \cup (A \times C)\text{.}
                \end{alignat*}
            \end{proof}
 
        %---------------Question 4---------------%
        \item Prove that, for all sets $A$, $B$, and $C$, $A \times (B \cap C) = (A \times B) \cap (A \times C)$.
            \begin{proof}
                Fix arbitrary sets $A$, $B$, and $C$. $A \times (B \cap C) = (A \times B) \cap (A \times C)$ means:
                \begin{align*}
                        \text{i. } &A \times (B \cap C) \subseteq (A \times B) \cap (A \times C) \\
                        \text{ii. } &(A \times B) \cap (A \times C) \subseteq A \times (B \cap C).
                    \end{align*}

                \begin{enumerate}[i.]
                    \item \lQuote $A \times (B \cap C) \subseteq (A \times B) \cap (A \times C)$" means:
                        \begin{align*}
                            \lQuote \text{For all } z \in A \times (B \cap C)\text{, } z \in (A \times B) \cap (A \times C)\text{."}
                        \end{align*}

                        To prove this, we must fix an arbitrary $z \in A \times (B \cap C)$.\ Then, $z = (x, y)$, $x \in A$ and $y \in (B \cap C)$. So, $y \in B$ and $y \in C$. Therefore, $(x \in A$ and $y \in B)$ and $(x \in A$ and $y \in C)$. Thus,
                        \begin{alignat*}{2}
                            &z = (x,y) \in (A \times B) \cap (A \times C) \text{ and} \\
                            &z \in (A \times B) \cap (A \times C)\text{.}
                        \end{alignat*}

                    \item \lQuote $(A \times B) \cap (A \times C) \subseteq A \times (B \cap C)$" means:
                        \begin{align*}
                            \lQuote \text{For all } z \in (A \times B) \cap (A \times C)\text{, } z \in A \times (B \cap C)\text{."}
                        \end{align*}

                        To prove this, we must fix an arbitrary $z \in (A \times B) \cap (A \times C)$. Therefore, $z \in (A \times B)$ and $z \in (A \times C)$. Then, $z = (x, y)$, $x \in A$ and $y \in B$ and $x \in A$ and $y \in C$. Since $y \in B$ and $y \in C$, $y \in (B \cap C)$. Since $x \in A$ and $y \in (B \cap C)$,
                        \begin{alignat*}{2}
                            &z = (x,y) \in A \times (B \cap C) \text{ and} \\
                            &z \in A \times (B \cap C)\text{.}
                        \end{alignat*}
                \end{enumerate}

                Therefore, since
                \begin{alignat*}{2}
                    &A \times (B \cap C) \subseteq (A \times B) \cap (A \times C)\text{ and} \\
                    &(A \times B) \cap (A \times C) \subseteq A \times (B \cap C)\text{, then} \\
                    &A \times (B \cap C) = (A \times B) \cap (A \times C)\text{.}
                \end{alignat*}
            \end{proof}

        %---------------Question 5---------------%
	\item Let $A = \{ 1, 2, 3, 4 \}$.  Which of the following sets are equivalence relations on $A$?
		\begin{align*}
			R = & \ \{ (1,2), (2,3), (3,4), (4,1) \}\\
			S = & \ \{ (1,1), (2,2), (3,3), (4,4) \}, \\
			T = & \ \{ (1,1), (1,2), (2,1), (2,2), (3,3) \}, \\
			U = & \ \{ (1,1), (1,3), (2,2), (2,4), (3,3), (4,4) \}, \\
			V = & \ \{ (1,1), (1,2), (1,3), (2,1), (2,2), (2,3), (3,1), \\
			    & \ \ \ \ (3,2), (3,3), (4,4) \}.
		\end{align*}
            $R$ is not an equivalence relation $(1 \text{\cancel{R}} 1)$. \\
            $S$ is an equivalence relation. \\
            $T$ is not an equivalence relation $(4 \text{\cancel{R}} 4)$. \\
            $U$ is not an equivalence relation $(2 \text{R} 4$ but $4 \text{\cancel{R}} 2)$. \\
            $V$ is an equivalence relation.

        %---------------Question 6---------------%
	\item For each definition of $\sim$, determine if $\sim$ is an equivalence relation on $\mathbb{Z}$.  Justify your answer.
		\begin{enumerate}

                %---------6a---------%
			\item For all $a, b \in \mathbb{Z}$, $a \sim b$ if $a + b$ is even.
                    \begin{proof} Fix arbitrary $a, b, c \in \mathbb{Z}$.
                        \begin{itemize}
                            \item \textbf{Reflexive:}  $a \sim a$ since $a + a = 2a$, which is even.
                            \item \textbf{Symmetric:}  If $a \sim b$, then $a + b$ is even. This implies that $b + a$ is even. So, $b \sim a$.
                            \item \textbf{Transitive:} If $a \sim b$ and $b \sim c$, then let $a + b = 2k$ and $b + c = 2r$, $k, r\in \mathbb{Z}$. Adding them together,
                            \begin{alignat*}{2}
                                a + 2b + c &= 2k + 2r \\
                                a + c &= 2k + 2r - 2b \\
                                a + c &= 2\cdot(k + r - c) \\
                                &= 2\cdot l\text{,}&\qquad \textrm{Letting $l = (k + r - c)\in\mathbb{Z}$}
                            \end{alignat*}
                            So, $a + c$ is even. Thus, $a \sim c$.
                        \end{itemize}
                        Since this relation is reflexive, symmetric, and transitive on $\mathbb{Z}$, this is an equivalence relation on $\mathbb{Z}$.
                    \end{proof}

                %---------6b---------%
			\item For all $a, b \in \mathbb{Z}$, $a \sim b$ if $a \cdot b$ is even.
                    \begin{proof} Fix arbitrary $a, b, c \in \mathbb{Z}$.
                        \begin{itemize}
                            \item \textbf{Reflexive:} Let $a = 5$. Then, $a \cancel{\sim} a$ since $a + a = 5 \cdot 5 = 25$, which is not even.
                        \end{itemize}
                        Since this relation is not reflexive on $\mathbb{Z}$, it is not an equivalence relation on $\mathbb{Z}$.
                    \end{proof}

                %---------6c---------%
			\item For all $a, b \in \mathbb{Z}$, $a \sim b$ if $|a - b| \le 2$.
                    \begin{proof} Fix arbitrary $a, b, c \in \mathbb{Z}$.
                        \begin{itemize}
                            \item \textbf{Reflexive:} $a \sim a$ since $|a - a| = |0| = 0 \le 2$.
                            \item \textbf{Symmetric:} If $a \sim b$, then $|a - b| \le 2$. Then, $|b - a| = |\minus(a - b)| = |a - b| \le 2$. Thus, $b \sim a$. 
                            \item \textbf{Transitive:} Let $a = 5$, $b = 3$, and $c = 1$. Then, $a \cancel{\sim} c$ since $5 \sim 3$ and $3 \sim 1$, but $5$ $\cancel{\sim}$ $1$.
                        \end{itemize}
                        Since this relation is not transitive on $\mathbb{Z}$, it is not an equivalence relation on $\mathbb{Z}$.
                    \end{proof}

                %---------6d---------%
			\item For all $a, b \in \mathbb{Z}$, $a \sim b$ if $a - b$ is a multiple of $3$.
                    \begin{proof} Fix arbitrary $a, b, c \in \mathbb{Z}$.
                        \begin{itemize}
                            \item \textbf{Reflexive:} $a \sim a$ since $a - a = 0$ and $0 = 3 \cdot 0$.
                            \item \textbf{Symmetric:} If $a \sim b$, then $a - b = 3\cdot k$, $k\in\mathbb{Z}$. Then, 
                                \begin{alignat*}{2}
                                    b - a &= \minus(\minus b + a) = \minus(a - b) \\
                                    &= \minus(3\cdot k) \\
                                    &= 3 \cdot (\minus k) \\
                                    &= 3 \cdot l\text{,}\qquad &\textrm{Letting $l = (\minus k)\in\mathbb{Z}$}
                                \end{alignat*}
                                So, $b \sim a$.
                                \item \textbf{Transitive:} If $a \sim b$ and $b \sim c$, then let $a - b = 3\cdot k$ and $b - c = 3\cdot l$, $k, l\in\mathbb{Z}$. Adding them together, 
                                \begin{alignat*}{2}
                                    a - b + (b - c) &= a - c \\
                                    &= 3\cdot k + 3\cdot l \\
                                    &= 3\cdot (k + l) \\
                                    &= 3\cdot m\text{,}\qquad &\textrm{Letting $m = (k + l)\in\mathbb{Z}$}
                                \end{alignat*}
                                So, $a - c$ is a multiple of 3. Thus, $a \sim c$.
                        \end{itemize}
                        Since this relation is reflexive, symmetric, and transitive on $\mathbb{Z}$, this is an equivalence relation on $\mathbb{Z}$.
                    \end{proof}
		\end{enumerate}
    \end{enumerate}
    
    \newpage
    \textbf{Bonus Questions:}
    
    \begin{itemize}

        %---------------Question B1---------------%
        \item [(B1)] Prove that the Kuratowski ordered pair works: For all objects $a$, $b$, $c$, and $d$,
    	\[
    		\{ \{ a \}, \{ a, b \} \} = \{ \{ c \}, \{ c, d \} \} \text{ if and only if } a = c \text{ and } b = d.
    	\]
        I'm going to come into office hours to ask a question about this :(

        %---------------Question B2---------------%
        \item [(B2)] How many equivalence relations are there on $\{ 1, 2, 3 \}$?
            \begin{enumerate}
                \item $\{ (1, 1), (2, 2), (3, 3) \}$
                \item $\{ (1, 1), (1, 2), (2, 1), (2, 2), (3, 3) \}$
                \item $\{ (1, 1), (2, 2), (2, 3), (3, 2), (3, 3) \}$
                \item $\{ (1, 1), (1, 3), (2, 2), (3, 2), (3, 3) \}$
                \item $\{ (1, 1), (1, 2), (1, 3), (2, 1), (2, 2), (2, 3), (3, 1), (3, 2), (3, 3) \}$
            \end{enumerate}
            There are 5 equivalence relations on $\{ 1, 2, 3 \}$.
    \end{itemize}

\end{document}