\documentclass{article}
\usepackage{267_Lib}

\title{MATH 267 - Homework 7}
\author{Sele Okojie}
\date{March 27, 2023}

\begin{document}
    \maketitle

    \begin{enumerate}

        %---------------Question 1---------------%
	\item Prove that the rationals are dense in themselves.
            \begin{proof}
                Fix arbitrary $a, b\in\mathbb{Q}$ where $a < b$. Since the rationals are closed under addition and multiplication, we can find a rational number $c = \frac{a + b}{2}$. Note that $a < c < b$. Therefore, by definition, the set of rationals are dense in themselves.
            \end{proof}

        %---------------Question 2---------------%
	\item Prove that the integers are complete.
            \begin{proof}
                To prove that the set of integers are complete, we need to show that if a subset $B$ of the integers is bounded above, then its supremum belongs to the set of integers. 
                \ppar Let $B \subseteq \mathbb{Z}$ be bounded above. Thus, there exists some $m\in\mathbb{R}$ such that, $b \le m$ for all for all $b\in B$. We need to show that $B$ has a supremum in the integers. To do so, we let $S$ be the set of all upper bounds of $B$. Since $B$ is bounded above, $S$ is non-empty. Then, let $s$ be the supremum of $S$ in the real numbers. $s$ exists because $S$ is bounded above. We will show that $s$ is also the supremum of $B$ in the integers.
                
                \ppar Let $b\in B$. Then, since $s$ is an upper bound of $S$ and $B\subseteq S$, $b \le s$. Thus, $s$ is an upper bound of $B$. Then, suppose that there exists a $t \in\mathbb{Z}$ where $t < s$ and $t$ is an upper bound of B. Then, since $t < s$, there exists some $x \in S$ where $t < x \le s$. However, since $x$ is an upper bound of $B$, then $t$ is not an upper bound of $S$, contradicting the fact that $s$ is the supremum of $S$. Therefore, we reject our assumption and conclude that $s$ is the least upper bound of $B$ in the integers.
                
                \ppar Finally, since $s$ is an upper bound of $B$ and it is the least upper bound of $B$ in the integers, $s$ is the supremum of $B$ in the integers, and the integers are complete.
            \end{proof}
        
        %---------------Question 3---------------%
	\item For each of the following, determine if it is a function or not.
		\begin{align*}
			f = & \ \{ (1, 2), (2, 3), (3, 4), (4, 5) \} \\
			g = & \ \{ (1, 2), (1, 3), (1, 4), (1, 5) \} \\
			h = & \ \{ (1, 5), (2, 5), (3, 5), (4, 5) \} \\
			i = & \ \{ (x,x) : x \in \mathbb{Z} \} \\
			j = & \ \{ (x,y) : x, y \in \mathbb{R} \text{ and } x < y \} \\
			k = & \ \{ (x,y) : x, y \in \mathbb{R} \text{ and } x + y = 17 \}
		\end{align*}
            \begin{itemize}
                \item $f$ is a function.
                \item $g$ is not a function because $(1, 2), (1, 3) \in g$.
                \item $h$ is a function.
                \item $i$ is a function.
                \item $j$ is not a function because $(1, 2), (1, 3) \in g$.
                \item $k$ is a function.
            \end{itemize}

        %---------------Question 4---------------%
	\item For each function in (3), find its domain and image.
            \begin{itemize}
                \item dom($f$) = $\{ 1, 2, 3, 4 \}$, im($f$) = $\{ 2, 3, 4, 5 \}$.
                \item dom($h$) = $\{ 1, 2, 3, 4 \}$, im($h$) = $\{ 5 \}$.
                \item dom($i$) = $\mathbb{Z}$, im($i$) = $\mathbb{Z}$.
                \item dom($k$) = $\mathbb{R}$, im($k$) = $\{ (x, 17 - x) : x\in \mathbb{R} \}$.
            \end{itemize}

        %---------------Question 5---------------%
	\item Determine if the following is a well-defined function $f : \mathbb{Q} \rightarrow \mathbb{Q}$.  Justify your answer.
		\[
			f \left( \frac{p}{q} \right) = \frac{ p + q }{ q^3 }.
		\]
            $f$ is not a well-defined function.
            \begin{proof}
                Consider $\frac{4}{2} = \frac{2}{1}$. Then, 
                \begin{align*}
                    f \left( \frac{4}{2} \right) &= \frac{4 + 2}{2^3} = \frac{6}{8} = \frac{3}{4}.\\ 
                    f \left( \frac{2}{1} \right) &= \frac{2 + 1}{1^3} = \frac{3}{1} = 3.\\ 
                \end{align*}
                Thus, $f$ is not well-defined.
            \end{proof}

        %---------------Question 6---------------%
	\item Determine if the following is a well-defined function $f : \mathbb{Q} \rightarrow \mathbb{Q}$.  Justify your answer.
		\[
			f \left( \frac{p}{q} \right) = \frac{ 2p + q }{ 3q }.
		\]
            $f$ is a well-defined function.
            \begin{proof}
                Fix rational number $\frac{r}{s} = \frac{p}{q}$, where $p, q, r, s \in\mathbb{Z}$ and $q, s \ne 0$. Then, 
                \begin{align*}
                    f \left( \frac{p}{q} \right) &= \frac{ 2p + q }{ 3q } \\
                    &= \frac{2p}{3q} + \frac{q}{3q} \\
                    &= \frac{2}{3} \cdot \frac{p}{q} + \frac{1}{3} \\
                    &= \frac{2}{3} \cdot \frac{r}{s} + \frac{s}{3s} \\
                    &= \frac{2r}{3s} + \frac{s}{3s} \\
                    &= \frac{2r + s}{3s} =  f \left( \frac{r}{s} \right).
                \end{align*}
                Thus, $f (\frac{p}{q}) = f(\frac{r}{s})$ and since $\frac{p}{q} = \frac{r}{s}$, $f$ is well-defined.
            \end{proof}
        
        %---------------Question 7---------------%
	\item For each function below, determine its image.
		\begin{align*}
			& f : \mathbb{R} \rightarrow \mathbb{R} & & f(x) = x^2 \\
			& g : (\mathbb{R} \setminus \{ 3 \}) \rightarrow \mathbb{R} & & g(x) = \frac{2x + 5}{x - 3} \\
			& h : \mathbb{R}^2 \rightarrow \mathbb{R} & & h((x,y)) = x - y \\
			& i : \mathbb{R} \rightarrow \mathbb{R}^2 & & i(x) = (x, 2x) 
		\end{align*}
             \begin{itemize}
                \item im($f$) = $\mathbb{R}_{\ge 0}$.
                \item im($g$) = $\{ \mathbb{R} \setminus \{ 2 \} \}$.
                \item im($h$) = $\mathbb{R}$.
                \item im($h$) = $\{ (x, 2x) : x \in\mathbb{R} \}$.
            \end{itemize}

        %---------------Question 8---------------%
	\item Justify your answer for $g$ in question (7).
            \begin{proof}
                To find the image of $g$, we need to determine all the possible values that $g(x)$ can take for any $x$ in the domain.
                
                \ppar Note that as $x$ approaches $3$ from either side, $g(x)$ diverges to $\infty^+$ or $\infty^\minus$, depending on the sign of $(2x + 5)$. We must also note what happens as $x$ goes to infinity. For $\infty^+$, the numerator grows faster than the denominator, so $g(x)$ approaches $2$. Similarly, for $\infty^\minus$, the numerator grows faster than the denominator, so $g(x)$ approaches $\minus2$.
                
                \ppar Thus, we can conclude that the image of $g$ is the set of all real numbers except $2$, or im($g$) = $\{ \mathbb{R} \setminus \{ 2 \} \}$.
            \end{proof}
 
    \end{enumerate}

    \newpage
    \textbf{Bonus Questions:}
    
    \begin{itemize}

        %---------------Question B1---------------%
        \item [(B1)] How many functions are there with domain $\{ 1, 2, 3, 4 \}$ and codomain $\{ 1, 2, 3 \}$? \par
            \qquad For each of the four elements in the domain, we have three possible choices for the value it is mapped to. Therefore, the total number of functions are $3 \cdot 3 \cdot 3 \cdot 3 = 81$.

        %---------------Question B2---------------%
        \item [(B2)] Give an example of a function $f : \mathbb{R} \rightarrow \mathbb{R}^2$ whose image is the unit circle.  In other words, $\mathrm{im}(f) = \left\{ (x,y) : x, y \in \mathbb{R} \text{ and } x^2 + y^2 = 1 \right\}$ = ?
            \begin{align*}
                f(t) = (\cos{t}, \sin{t}).
            \end{align*}
    \end{itemize}

\end{document}