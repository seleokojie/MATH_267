\documentclass{article}
\usepackage{267_Lib}

\title{MATH 267 - Homework 9}
\author{Sele Okojie}
\date{April 10, 2023}

\begin{document}
    \maketitle

    \begin{enumerate}

        %---------------Question 1---------------%
        \item Prove that, for all sets $A$, $\mathrm{id}_A \circ \mathrm{id}_A = \mathrm{id}_A$.
            \begin{proof}
                Fix an arbitrary set $A$ and an arbitrary $a\in A$. Then,
                \[
                    (\mathrm{id}_A \circ \mathrm{id}_A)(a) = \mathrm{id}_A(\mathrm{id}_A(a)) = \mathrm{id}_A(a).
                \]
            \end{proof}

        %---------------Question 2---------------%
	\item Let $A = \{ 1, 2, 3 \}$, $B = \{ 4, 5, 6 \}$, and $f : A \rightarrow B$ be given by
		\[
			f(1) = 5, \ f(2) = 5, \ f(3) = 6.
		\]
		Compute each of the following images and preimages:
		\begin{align*}
			\text{(a)} & \ \ f(\{ 1 \}) = \{ 5 \} & \text{(b)} & \ \ f(\{ 1, 2 \}) = \{ 5 \} \\
			\text{(c)} & \ \ f( \emptyset ) = \emptyset & \text{(d)} & \ \ f^{\minus1}(\{ 4 \}) = \emptyset \\
			\text{(e)} & \ \ f^{\minus1}(\{ 5 \}) = \{ 1, 2 \} & \text{(f)} & \ \ f^{\minus1}(\{ 4, 6 \}) = \{ 3 \}
		\end{align*}

        %---------------Question 3---------------%
	\item Let $f : \mathbb{R} \rightarrow \mathbb{R}$ be given by
		\[
			f(x) = x^3 - 3x.
		\]
		Compute each of the following images and preimages:
		\begin{align*}
			\text{(a)} & \ \ f(\{0, 1\}) = \{ 0, \minus2 \} & \text{(b)} & \ \ f([0,1]) = [\minus2, 0] \\
			\text{(c)} & \ \ f(\emptyset) = \emptyset & \text{(d)} & \ \ f^{\minus1}(\{ 2 \}) = \{ \minus1, 2 \} \\
			\text{(e)} & \ \ f^{\minus1}([-2,2]) = [\minus2, 2] & \text{(f)} & \ \ f^{\minus1}(\emptyset) = \emptyset
		\end{align*}

        %---------------Question 4---------------%
	\item Let $f : \mathbb{R} \rightarrow \mathbb{R}^2$ be given by
		\[
			f(x) = (\cos(x), \sin(x)).
		\]
		Draw each of the following images in the Euclidean plane:
		\begin{align*}
			\text{(a)} & \ \ f(\{ 0 \}) & \text{(b)} & \ \ f([0, \pi])
		\end{align*}
            \begin{tikzpicture}
                \begin{axis}[
                    axis on top,
                    axis lines=center,
                    axis line style={latex-latex}, % add arrow tips
                    unit vector ratio*=1 1 1,
                    xlabel=$x$,
                    ylabel=$y$,
                    ymin=0, ymax=1,
                    xmin=-1, xmax=1,
                    xtick={-1,-0.5,0,0.5,1},
                    ytick={-1,-0.5,0,0.5,1},
                    ticklabel style={font=\small},
                    enlargelimits={abs=0.2},
                    clip=false
                ]
                    \addplot [only marks, red, mark=*] coordinates {(1,0)};
                    \node [above left, red] at (axis cs:1,0) {$f(\{0\})$};
                    
                    \addplot [domain=0:pi, samples=100, blue, thick, name path=A] ({cos(deg(x))},{sin(deg(x))});
                    \addplot [name path=B] coordinates {(-1,0) (1,0)};
                    %\addplot [blue!30] fill between[of=A and B, soft clip={domain=-1:1}]; %This is how to shade under a graph for future use
                    \node [above right, blue] at (axis cs:0.7,0.7) {$f([0,\pi])$};
                \end{axis}
            \end{tikzpicture}

        %---------------Question 5---------------%
	\item \emph{Proposition 10.2.1 (2)}: Prove that, for all sets $A$ and $B$, for all $C_1, C_2 \subseteq A$, and for all functions $f : A \rightarrow B$,
		\[
			f(C_1 \cap C_2) \subseteq f(C_1) \cap f(C_2).
		\]
            \begin{proof} Fix some $b\in f(C_1 \cap C_2)$. So, there exists some $a\in C_1 \cap C_2$ such that $b = f(a)$. So, $a\in C_1$ and $a\in C_2$. Then, if $a\in C_1$, then $b\in f(C_1)$. Similarly, if $a\in C_2$, then $b\in f(C_2)$. Thus, $b\in f(C_1) \cap f(C_2)$. Therefore, $f(C_1 \cap C_2) \subseteq f(C_1) \cap f(C_2)$.
            \end{proof}

        %---------------Question 6---------------%
	\item \emph{Proposition 10.2.1 (8)}: Prove that, for all sets $A$ and $B$, for all $D \subseteq B$, and for all functions $f : A \rightarrow B$,
		\[
			f(f^{\minus1}(D)) \subseteq D.
		\]
            \begin{proof}
                Fix some $a\in f(f^{\minus1}(D))$. By the definition of pre-image, there exists some $b\in f^{\minus1}(D)$ such that $f(b) = a$. Since $b\in f^{\minus1}(D)$,
                then $f(b)\in D$. Thus, $a = f(b)\in D$, proving that $f(f^{\minus1}(D)) \subseteq D$.
            \end{proof}
    \end{enumerate}

\newpage
\textbf{Bonus Questions:}

\begin{itemize}

        %---------------Question B1---------------%
	\item [(B1)] \emph{What assumptions do we really need to prove Theorem 10.1.3?}  Assume that $X$ is a set and $*$ and ${}^{\minus1}$ are operations on $X$ such that,
		\begin{itemize}
			\item for all $a, b \in X$, $a * b \in X$,
			\item for all $a \in X$, $a^{\minus1} \in X$,
			\item for all $a, b, c \in X$, $a * (b * c) = (a * b) * c$,
			\item there exists $e \in X$ such that $a * e = e * a = a$,
			\item for all $a \in X$, $a * a^{\minus1} = a^{\minus1} * a = e$.
		\end{itemize}
        Prove that, for all $a, b \in X$, $(a * b)^{-1} = b^{-1} * a^{-1}$.
            \begin{proof}
                We need to show that:
                    \[
                    (a * b) * (b^{\minus1} * a^{\minus1}) = e \text{ and } 
                    (b^{\minus1} * a^{\minus1}) * (a * b) = e.
                    \]
                    Note that inverses are unique.
                \begin{caseof}
                    \case{($\implies$)}{Using the third property,
                        \[
                            (a * b) * (b^{\minus1} * a^{\minus1}) = a * (b * (b^{\minus1} * a^{\minus1})).
                        \]
                        Since $b * b^{\minus1} = e$, we can simplify the equation into:
                        \[
                            a * (e * a^{\minus1}) = a * a^{\minus1} = e.
                        \]
                        Therefore, $(a * b) * (b^{\minus1} * a^{\minus1}) = e$.
                    }\case{($\impliedby$)}{Using the third property,
                        \[
                            (b^{\minus1} * a^{\minus1}) * (a * b) = b^{\minus1} * (a^{\minus1} * a) * b.
                        \]
                        Since $a * a^{\minus1} = e$, we can simplify the equation into:
                        \[
                            b^{\minus1} * e * b = b^{\minus1} * b = e.
                        \]
                        Therefore, $(b^{\minus1} * a^{\minus1}) * (a * b) = e$.
                    }
                \end{caseof}
                Thus, we can conclude that $(a * b)^{-1} = b^{-1} * a^{-1}$ for all $a, b \in X$.
            \end{proof}

        %---------------Question B2---------------%
	\item [(B2)] Consider the function $f : \mathbb{R}^2 \rightarrow \mathbb{R}$ given by
		\[
			f((x,y)) = x + y.
		\]
		Compute $f^{-1}( \{ 0 \} )$ and justify your answer.
            \begin{proof}
                To compute $f^{-1}({0})$, we need to find all points in $\mathbb{R}^2$ that are mapped to 0 under $f$. To find these points, we want to solve the equation $x+y=0$ to find all points $(x,y)$ that satisfy $f((x,y))=0$.
                This equation represents the line $y=-x$ in $\mathbb{R}^2$. So, $f^{-1}({0})$ is the set of all points on the line $y=-x$. Substituting this into the original function,
                \[
                    f((x, y)) = x + y = x + (\minus x) = 0.
                \]

To summarize, we have $f^{-1}({0}) = {(x,y) \in \mathbb{R}^2 : x+y = 0}$, which is the line $y=-x$.
            \end{proof}
\end{itemize}

\end{document}