\documentclass{article}
\usepackage{267_Lib}

\title{MATH 267 - Homework 11}
\author{Sele Okojie}
\date{April 24, 2023}

\begin{document}
    \maketitle
    
    \begin{enumerate}

            %---------------Question 1---------------%
    	\item Does $0 \mid 0$?  Justify your answer.
                \begin{proof}
                    Since there exists an integer $k$ such that $0 = k \cdot 0$, then $0 \mid 0$.
                \end{proof}

            %---------------Question 2---------------%
    	\item Prove that, for all integers $a$ and $n$,
    		\[
    			n \mid a \text{ if and only if } a \equiv 0\ (\mathrm{mod}\ n).
    		\]
                \begin{proof}
                    Let $a, n\in\mathbb{Z}$. We want to prove that 
                    \begin{align*}
                        \text{i. }   & \text{If $n \mid a$, then $a \equiv 0\ (\mathrm{mod}\ n)$ and}  \\
                        \text{ii. }  & \text{If $a \equiv 0\ (\mathrm{mod}\ n)$, then $n \mid a$.}
                    \end{align*}
                                
                    \begin{enumerate}[i.]
                        \item Assume that $n \mid a$. Then, there exists some $k\in\mathbb{Z}$ such that $a = n\cdot k$. Subtracting $0$ from both sides, we get $a - 0 = n\cdot k$. Since $n\cdot k\in\mathbb{Z}$, this means that $n \mid (a - 0)$, which means that $a \equiv 0\ (\mathrm{mod}\  n)$. \\
                                    
                        \item Assume that $a \equiv 0\ (\mathrm{mod}\ n)$. Then, by the definition of congruence, $n \mid (a - 0)$. Then, since $(a - 0) = a$, we can say that $n \mid a$.
                    \end{enumerate}
                    Since we have proved both statements, we can conclude that $n \mid a$ iff $a \equiv 0\ (\mathrm{mod}\ n)$, for all integers $a, n$.
                \end{proof}

            %---------------Question 3---------------%
    	\item Prove that, for all integers $a$, $b$, $c$, $d$, $n$, and $m$,
    		\[
    			\text{if } a \mid b \text{ and } a \mid c, \text{ then } a \mid (bn + cm).
    		\]
                \begin{proof}
                    Let $a, b, c, d, n\in\mathbb{Z}$ and assume that $a \mid b$ and $a \mid c$. Then, since $a \mid b$, there exists some $k_1\in\mathbb{Z}$ such that $b = a\cdot k_1$. Similarly, since $a \mid c$, there exists some $k_2\in\mathbb{Z}$ such that $c = a\cdot k_2$.
                    \ppar Then, using these new expressions for $b$ and $c$,
                        \begin{align*}
                            bn + cm &= (a\cdot k_1)\cdot n + (a \cdot k_2)\cdot m \\
                            &= a \cdot (k_1\cdot n + k_2 \cdot m).
                        \end{align*}
                    \ppar Thus, since $k_1, k_2, n, m\in\mathbb{Z}$, then $(k_1\cdot n + k_2 \cdot m)\in\mathbb{Z}$. This means that $a \mid (bn + cm)$. Thus, for all integers $a$, $b$, $c$, $d$, $n$, and $m$, if $a \mid b$ and $a \mid c$, then $a \mid (bn + cm)$.
                \end{proof}

            %---------------Question 4---------------%
    	\item Prove that, for all integers $a$,
    		\[
    			3 \mid a \text{ if and only if } 3 \mid a^2.
    		\]
                \begin{proof}
                    Let $a\in\mathbb{Z}$. We want to prove that 
                    \begin{align*}
                        \text{i. }   & \text{If $3 \mid a$, then $3 \mid a^2$ and}  \\
                        \text{ii. }  & \text{If $3 \mid a^2$, then $3 \mid a$.}
                    \end{align*}
                                
                    \begin{enumerate}[i.]
                        \item Assume that $3 \mid a$. Then, by the definition of divisibility, there exists some $k\in\mathbb{Z}$ such that $a = 3\cdot k$. Squaring both sides of the equation,
                            \[
                                a^2 = (3\cdot k)^2 = 9\cdot k^2.
                            \]
                        \ppar Since $9\cdot k^2 = 3\cdot(3\cdot k^2)$, there exists an integer $m = 3\cdot k^2$ such that $a^2 = 3\cdot m$. This implies that $3 \mid a^2$. \\
                        \item Assume that $3 \mid a^2$. Then, by the definition of divisibility, there exists some $k\in\mathbb{Z}$ such that $a^2 = 3\cdot k$.
                            \begin{subproof}[Subproof (Euclid's Lemma)]
                                Let $a,b\in\mathbb{Z}$. Furthermore, let $p$ be a prime number and assume that $p \mid a\cdot b$. We want to prove that either $p\ | \ a$ or $p\ | \ b$.
                                \begin{caseof}
                                    \case{($p\ | \ a$)}{
                                        If $p\ | \ a$, then we conclude. 

                                    }\case{($p\nmid|\ a$)}{
                                        Since $p$ is prime, the only positive divisors of $p$ are $1$ and $p$. However, $p$ does not divide $a$, so $\gcd(p, a) = 1$. This means that there exists $x, y\in\mathbb{Z}$ such that $px + ay = 1$. Multiplying both sides by $b$, we get:
                                        \[
                                            pxb + ayb = b
                                        \]
                                        Then, since $p \mid pxb$ and $p \mid ayb$ (because $p \mid a\cdot b$ and $p \mid p\cdot a$), 
                                        \[
                                            p \mid (pxb + ayb)
                                        \]
                                        Then, since $pxb + ayb = b$ by the previous equation, $p \mid b$. 
                                    }
                                \end{caseof}
                            \end{subproof}
                        Then, since $3$ is prime and $3 \mid a^2 = a\cdot a$, either $3 \mid a$ or $3 \mid a$, by Euclid's Lemma. Thus, $3 \mid a$, since these conditions are equivalent. 
                    
                    \end{enumerate}
                    Since we have proved both statements, we can conclude that $3 \mid a$ iff $3 \mid a^2$, for all integers $a$.
                \end{proof}

            %---------------Question 5---------------%
    	\item \emph{Prove Proposition 2.1.7.} Use the Quotient-Remainder Theorem to prove that, for all integers $a$, $2\nmid|\ a$ if and only if there exists an integer $c$ such that $a = 2c + 1$.
                \begin{proof}
                    Let $a\in\mathbb{Z}$. We want to prove that 
                    \begin{align*}
                        \text{i. }   & \text{If $2\nmid|\ a$, then there exists an integer $c$ such that $a = 2c + 1$ and}  \\
                        \text{ii. }  & \text{If there exists an integer $c$ such that $a = 2c + 1$, then $2\nmid|\ a$.}
                    \end{align*}
                                
                    \begin{enumerate}[i.]
                        \item Assume that $2\nmid|\ a$. Then, by the Quotient-Remainder Theorem, there exists unique integers $c,r$ such that $a = 2\cdot c + r$ and $0 \le r < 2$. Note that, since $2\nmid|\ a$, $r \neq 0$. Thus, $r$ must be $1$ and we have $a = 2\cdot c + 1$. \\
                                    
                        \item Assume that there exists an integer $c$ such that $a = 2c + 1$. Then, since $c\in\mathbb{Z}$, it follows that $a$ is odd. Thus, by the definitions of odd and even integers, $2\nmid|\ a$. 
                    \end{enumerate}
                    Since we have proved both statements, we can conclude that $2\nmid|\ a$ iff there exists an integer $c$ such that $a = 2c + 1$., for all integers $a$.
                \end{proof}

            %---------------Question 6---------------%
    	\item What is $\gcd(0,5)$?  Justify your answer.
                $\gcd(0, 5) = 5$.
                \begin{proof}
                    By definition, $\gcd(0, 5)$ is the largest positive integer that divides both integers without a remainder. Since any non-zero integer divides $0$, the set of common divisors of $0$ and $5$ is the set of divisors of $5$: $\left\{ \pm1, \pm5 \right\}$. The largest positive integer in this set is $5$, so $\gcd(0, 5) = 5$.
                \end{proof}

            %---------------Question 7---------------%
    	\item Consider the numbers $1234$ and $4567$.
    		\begin{enumerate}
    			\item Use the Euclidean Algorithm to compute $\gcd(1234, 4567)$.
                        \begin{align*}
                            4567 &= 3 \cdot 1234 + 865 \\
                            1234 &= 1 \cdot 865  + 369 \\
                            865  &= 2 \cdot 369  + 127 \\
                            369  &= 2 \cdot 127  + 115 \\
                            127  &= 1 \cdot 115  + 12  \\
                            115  &= 9 \cdot 12   + 7   \\
                            12   &= 1 \cdot 7    + 5   \\
                            7    &= 1 \cdot 5    + 2   \\
                            5    &= 2 \cdot 2    + 1   \\
                            2    &= 2 \cdot 1    + 0   
                        \end{align*}
                        Since we reached a remainder of $0$, we will stop the algorithm. Thus, since the last non-zero remainder is $1$, $\gcd(1234, 4567) = 1$.
    			\item Find integers $n$ and $m$ such that
    				\[
    					1234n + 4567m = 1.
    				\]
                        Following our results from the Euclidean Algorithm,
                        \begin{align*}
                            1   &= 5    + (\minus2)\cdot 2    \\
                            2   &= 7    + (\minus1)\cdot 5    \\
                            5   &= 12   + (\minus1)\cdot 7    \\
                            7   &= 115  + (\minus0)\cdot 12   \\
                            12  &= 127  + (\minus1)\cdot 115  \\
                            115 &= 369  + (\minus2)\cdot 127  \\
                            127 &= 865  + (\minus2)\cdot 369  \\
                            369 &= 1234 + (\minus1)\cdot 865  \\
                            865 &= 4567 + (\minus3)\cdot 1234 
                        \end{align*}
                        Recursively substituting for \lQuote$5$" and \lQuote$2$" in the first equation, we will eventually get that 
                            \[
                                1 = 1234\cdot (\minus 1906) + 4567 \cdot (515)
                            \]
                        Thus, $n = \minus 1906$ and $m = 515$.
    			\item Find an integer $a$ such that
    				\[
    					1234a \equiv 5 \ (\mathrm{mod}\ 4567).
    				\]
                        Multiplying the resulting equation from $6b.$ by $5$,
                            \[
                                5 = 5 \cdot 1234\cdot (\minus 1906) + 5 \cdot 4567 \cdot (515)
                            \]
                        Since we are working in mod $4567$, we can ignore the second term on the right side of the equation. This gives us 
                            \begin{align*}
                                5 \cdot 1234 \cdot (\minus 1906) \equiv 5\ (\mathrm{mod}\ 4567) \\
                                1234 \cdot (\minus 1906 \cdot 5) \equiv 5\ (\mathrm{mod}\ 4567) \\
                                1234 \cdot \minus 9530 \equiv 5\ (\mathrm{mod}\ 4567).
                            \end{align*}
                        Thus, $a = \minus9530$ is an integer such that $1234a \equiv 5 \ (\mathrm{mod}\ 4567).$
    		\end{enumerate}

            %---------------Question 8---------------%
    	\item Use Modular Arithmetic to answer each of the following questions:
    		\begin{enumerate}
    			\item What day of the week is it $8^{1729}$ days after a Monday? \ppar
                        Since there are $7$ days in a week, we can work $(\mathrm{mod}\ 7)$ to find the day of the week $8^{1729}$ days after a Friday.
                            \[
                                8 \equiv 1\ (\mathrm{mod}\ 7), \text{ so }1 + 8^{1729} = 1 + (1^{1729}) = 1 + 1 = 2\ (\mathrm{mod}\ 7).
                            \]
                        So, it is day $2$ after a Monday, which means it is a Tuesday.
                        
    			\item What day of the week is it $6^{5678}$ days after a Friday? \ppar
                        Since there are $7$ days in a week, we can work $(\mathrm{mod}\ 7)$ to find the day of the week $6^{5678}$ days after a Friday.
                            \[
                                6 \equiv \minus1\ (\mathrm{mod}\ 7), \text{ so }5 + 6^{5678} = 5 + (\minus1^{5678}) = 5 + 1 = 6\ (\mathrm{mod}\ 7).
                            \]
                        So, it is day $6$ after a Monday, which means it is a Saturday.
                        
    			\item What time is it $23^{9999}$ hours after 6pm? \ppar
                        Since there are $24$ hours in a day, we can work $(\mathrm{mod}\ 24)$ to find what time it is $23^{9999}$ hours after 6pm.
                            \[
                                23 \equiv \minus1\ (\mathrm{mod}\ 24), \text{ so }18 + 23^{9999} = 18 + (\minus1^{9999}) = 18 - 1 = 17\ (\mathrm{mod}\ 24).
                            \]
                        So, it is 5pm.
                    
    			\item What time is it $5^{4500}$ hours after 3am? \ppar
                        Since there are $24$ hours in a day, we can work $(\mathrm{mod}\ 24)$ to find what time it is $5^{4500}$ hours after 3am. Furthermore, note that $5^{1000} = 25^{500}$.
                            \[
                                25 \equiv 1\ (\mathrm{mod}\ 24), \text{ so }3 + 5^{4500} = 3 + (1^{4500}) = 3 + 1 = 4\ (\mathrm{mod}\ 24).
                            \]
                        So, it is 4am.
       
    			\item What time is it $5^{10001}$ hours after 4pm? \ppar
                        Since there are $24$ hours in a day, we can work $(\mathrm{mod}\ 24)$ to find what time it is $5^{4500}$ hours after 3am. Furthermore, note that $5^{10001} = 25^{5000} \cdot 5$.
                            \[
                                25 \equiv 1\ (\mathrm{mod}\ 24), \text{ so }16 + 25^{5000} \cdot 5 = 16 + (1^{5000}) \cdot 5 = 16 + 1\cdot 5 = 21\ (\mathrm{mod}\ 24).
                            \]
                        So, it is 9pm.
    		\end{enumerate}
    \end{enumerate}
    
    \newpage
    \textbf{Bonus Questions:}
    
    \begin{itemize}

            %---------------Question B1---------------%
    	\item [(B1)] \emph{Euclid's Lemma.} Prove that, for all prime numbers $p$, for all integers $a$ and $b$,
    		\[
    			\text{if } p \mid ab, \text{ then } p \mid a \text{ or } p \mid b.
    		\]
    		(\textbf{Hint:} if $p$ is prime and $p\nmid|\ a$, then $\gcd(p,a) = 1$.)
                \ppar Solved in Question 4.

            %---------------Question B2---------------%
    	\item [(B2)] Use (B1) to prove that, for all prime numbers $p$, $\sqrt{p}$ is irrational.
                \begin{proof}
                    To prove that $\sqrt{p}$ is irrational for all prime numbers $p$, we will use Euclid's lemma, which states that for all prime numbers $p$, if $p \mid ab$, then $p \mid a$ or $p \mid b$.

                    Towards a contradiction, suppose that $\sqrt{p}$ is rational. Then, there exist $a, b\in\mathbb{Z}$ such that $\sqrt{p}=\frac{a}{b}$, where $b\neq0$. We may assume that $a$ and $b$ have no common factors, since we can always divide them by their greatest common divisor without changing the value of the fraction.

                    Squaring both sides, we get 
                    \[
                        p = \frac{a^2}{b^2}.
                    \]
                    Multiplying both sides by $b^2$, we get
                    \[
                        pb^2 = a^2.
                    \]
                    Since $p$ is a prime number, and $p \mid pb^2$, Euclid's lemma tells us that either $p \mid p$ or $p \mid b^2$. Since $p$ is prime, we know that $p$ does not divide $b$. Therefore, $p$ must divide $a^2$.\\

                    Since $p$ is prime, it follows from the Fundamental Theorem of Arithmetic that there exist integers $k$ and $c$ such that $a^2 = p^kc$. Since $p$ is prime, it cannot divide $c$ (otherwise, $a^2$ would have a larger power of $p$ in its prime factorization). Therefore, we have $p \mid a^2$ and, since $p$ is prime, $p \mid a$. This means that $a$ and $p$ share a common factor and, since $p$ is prime, it cannot have any other factors besides $1$ and $p$. Therefore, $a$ and $p$ are not coprime. However, this contradicts our assumption that $a$ and $b$ are coprime.

                    Thus, our assumption that $\sqrt{p}$ is rational must be false, and so we conclude that $\sqrt{p}$ is irrational for all prime numbers $p$.
                \end{proof}
    \end{itemize}
    
\end{document}